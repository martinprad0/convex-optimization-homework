% chktex-file 3 chktex-file 9 chktex-file 10 chktex-file 17 chktex-file 18 chktex-file 36
\section*{Exercise 2.}

\textbf{Güler: Ex 2.9 }

Consider the function $f(x,y) = x^2 + \alpha xy^2 + 2 y^4$. Show that all the parameter values except two, the origin $(0,0)$ is the only critical point of $f$.

\begin{enumerate}[label=\alph*)]
    \item Find the exceptional $\alpha$'s and show that $f$ has infinitely many critical points for these $\alpha$ values. Determine the nature of these critical points.
    \item Consider the values of $\alpha$'s for which the origin is the only critical point. For each $\alpha$, determine the nature of the critical point. Show that in some cases, the origin is a local minimum, but in other cases, it is a saddle point.
    \item Show that even when the origin is a saddle point, (0,0) is a local strict minimizer of $f$ on every line passing through the origin. In fact, show that, except for one line, the function $g(t) := f(td)$ satisfies $g'(0) = 0$ and $g''(0) > 0$.
\end{enumerate}

\subsection*{Solution Part (a)}
The derivative of $f$ is
\[ Df(x,y) = \left( \begin{array}{c}
    2x + \alpha y^2\\
    2\alpha x y + 8y^3
\end{array} \right). \]
By solving $\alpha$ in $Df(x,y) = 0$ we can find where $f$ has infinitely many critical points. Then, we get the following equations, for $x,y \neq 0$:
\[ \everymath{\displaystyle}
\arraycolsep=1.8pt\def\arraystretch{1.5}
\begin{array}{rcl}
    \alpha = \frac{-2x}{y^2}\\
    \alpha = \frac{-8y^3}{2x y}
\end{array} \iff \frac{-2x}{y^2} = \frac{-8y^3}{2x y} \]
\[ \iff x^2 = 2y^4 \]
\[ \iff y^2 = \pm \frac{x}{\sqrt{2}}. \]
Therefore, the only parameter values where there are infinitely many solutions are:
\[ \alpha = \frac{-2x}{y^2} =  \frac{-2x}{\pm x\cdot \sqrt{1/2}} = \pm 2\sqrt{2}.\]
When $\alpha$ is different from these values, the chain of equivalences we made before tells us that $Df (x,y) \neq 0$ when $x,y\neq 0$. Since $Df(0,0) = 0$ for any $\alpha$, it follows that $(0,0)$ is the only critical point whenever $\alpha \neq \pm 2\sqrt{2}$.

\subsection*{Solution Part (b)}

The Hessian matrix is degenerate at (0,0), so I looked at Geogebra for a hint of the answer, it seems that when $\alpha \in [-2\sqrt{2}, 2\sqrt{2}]$, it is a minimum and a saddle point at $ \alpha \in (-\infty, -2\sqrt{2}) \cup (2\sqrt{2}, \infty)$.
\[ \left[ \begin{matrix}
    2 & 2\alpha y\\
    2\alpha y & 2\alpha x + 24 y^2
\end{matrix} \right]\Big|_{(0,0)} = \left[ \begin{matrix}
    2 & 0\\
    0 & 0
\end{matrix} \right] \]

To justify that, we must complete a square in the previous expression to show that for every $(x,y) \neq (0,0)$ the function doesn't decrease:
\[ \everymath{\displaystyle}
\arraycolsep=1.8pt\def\arraystretch{2.5}
\begin{array}{rcl}
    x^2 + \alpha xy^2 + 2 y^4 & = & \left( x^2 + \alpha xy^2 + \frac{1}{4}\alpha^2 y^4 \right) - \frac{1}{4}\alpha^2 y^4 + 2y^4\\
    & = & \left( x+ \frac{\alpha y^2}{2} \right)^2 + y^4 \left( 2- \frac{\alpha^2}{4} \right)
\end{array} \]
If $|\alpha| \leq 2\sqrt{2}$, then $\frac{\alpha^2}{4} \leq 2$. Therefore,
\[ y^4 \left( 2- \frac{\alpha^2}{4} \right) \geq 0, \]
and if $x,y \neq 0$, then
\[ \left( x+ \frac{\alpha y^2}{2} \right)^2 \geq 0. \]
It follows that $(0,0)$ is a global minimum, since every slight deviation from the origin will make $f$ stay the same or increase. Now, for the case when $|\alpha| > 2 \sqrt{2}$ the oposite happens when we consider the following curve,
\[ x(t) = -\frac{\alpha t^2}{2},\hspace{1em} y(t) = t. \]
If $|\alpha| > 2\sqrt{2}$, then $\frac{\alpha^2}{4} > 2$. Therefore, when $t \neq 0$
\[ t^4 \left( 2- \frac{\alpha^2}{4} \right) < 0, \]
and
\[ \left( x(t) + \frac{\alpha y(t)^2}{2} \right)^2 = \left( -\frac{\alpha t^2}{2} + \frac{\alpha t^2}{2} \right)^2 = 0. \]
Therefore, function evaluated on the curve $(x(t), y(t))$ decreases when $t \neq 0$.
\subsection*{Solution Part (c)}

Let $d = (x_0,y_0)$. Then
\[ g(t) = f(tx_0, t y_0) = t^2 x_0^2 + \alpha t^3 x_0 y_0^2 + 2 t^4 y_0^4, \]
\[ g'(t) = 2t x_0^2 + 3\alpha t^2 x_0 y_0^2 + 8 t^3 y_0^4, \]
\[ g''(t) = 2x_0^2 + 6 \alpha t x_0 y_0^2 + 24 t^2 y_0^4. \]
It's clear that $g'(0) = 0$ for every $d \in \R^2\backslash\{\vec{0}\}$ and $g''(0) = 2 x_0^2 > 0$ whenever $x_0 \neq 0$. Therefore, we know for sure that $t = 0$ minimizes $g$ in every line except for $(x(t), y(t)) = (0,t)$.

